\documentclass{article}
\usepackage{graphicx} % Required for inserting images

\title{\textbf{Game-Based Intervention for ADHD}}
\author{Abhinav Shripad (2022CS11596) \\ Jahnabi Roy (2022CS11094)}
\date{April 2024}

\begin{document}

\maketitle

\section{Introduction}
Attention Deficit Hyperactivity Disorder (ADHD) is a neurodevelopmental disorder characterized by difficulty sustaining attention, hyperactivity, and impulsive behavior. It can significantly impact an individual's daily functioning, including academic performance and social interactions. As such, finding effective interventions to manage ADHD is crucial.

In this report, we present a game-based intervention designed to assist individuals diagnosed with ADHD. The intervention utilizes a series of simple games aimed at improving attention, memory, and cognitive flexibility. By engaging individuals in enjoyable activities, the intervention aims to enhance their cognitive abilities and reduce ADHD-related symptoms.

Additionally, we have integrated a questionnaire about ADHD into the web-based game platform. This questionnaire allows us to gather valuable information about participants' ADHD symptoms, severity, and any coexisting conditions, enabling us to tailor the intervention to their specific needs and monitor their progress effectively.

\section{Games Used}
\subsection{1-15 Number Puzzle}
The 1-15 number puzzle, also known as the sliding puzzle, is a classic game where players rearrange numbered tiles on a grid to form a sequential order. This game promotes problem-solving skills, spatial reasoning, and concentration. By requiring players to focus on the task at hand and think strategically, it can help improve attention and executive functions in individuals with ADHD.

\subsection{Memory Card Game}
The memory card game is a popular cognitive exercise where players match pairs of cards by remembering their locations on a grid. This game challenges working memory, attention to detail, and visual processing skills. Individuals with ADHD often struggle with working memory tasks, making this game particularly beneficial for enhancing their memory and concentration abilities.

\subsection{8 Queens Game}
The 8 Queens game is a puzzle that involves placing eight chess queens on an 8×8 chessboard in a way that no two queens threaten each other. This game requires careful planning, problem-solving, and spatial awareness. By engaging individuals in logical thinking and strategic planning, it can help improve their cognitive flexibility and impulse control, both of which are areas of difficulty for individuals with ADHD.

\subsection{Tower of Hanoi Game}
The Tower of Hanoi is a mathematical puzzle where players move a stack of disks from one rod to another, following specific rules. This game challenges spatial reasoning, planning, and sequential processing skills. By encouraging individuals to break down complex tasks into smaller steps and think ahead, it can help improve their organizational skills and attention span, which are often impaired in individuals with ADHD.

\section{Scoring and Participant Assessment}
Each game in the intervention is scored based on various metrics such as completion time, accuracy, and number of moves. Participants' performance is assessed by comparing their scores to established benchmarks or norms for their age group. Additionally, qualitative observations regarding participants' engagement, frustration levels, and strategies employed are recorded to provide a comprehensive assessment of their abilities and progress over time.

\section{Conclusion}
ADHD is a complex disorder that requires comprehensive interventions to address its multifaceted symptoms. Game-based interventions offer a promising approach to improving cognitive abilities and reducing ADHD-related difficulties. By incorporating engaging and stimulating games into therapy or educational programs, individuals with ADHD can develop essential skills while enjoying themselves. However, further research is needed to evaluate the efficacy of game-based interventions and optimize their implementation for individuals with ADHD.

\end{document}
`'